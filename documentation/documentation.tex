\documentclass[a4paper, 12pt]{article}
\usepackage[margin = 2cm]{geometry}

\usepackage[utf8]{inputenc}
\usepackage[OT2, T1]{fontenc}
\usepackage[english, serbian]{babel}

\usepackage{amsfonts, amssymb, amsmath}
\usepackage{graphicx, float, enumitem, hyperref}

\usepackage{listings}          % for creating language style
\input{arduinoLanguage.tex}    % adds the arduino language listing

%% Define an Arduino style fore use later %%
\lstdefinestyle{Arduino}{
  language=Arduino,
  %% Add other words needing highlighting below %%
  morekeywords=[1]{},                  % [1] -> dark green
  morekeywords=[2]{Keypad, LiquidCrystal_I2C, DHT22},        % [2] -> light blue
  morekeywords=[3]{},  % [3] -> bold orange
  morekeywords=[4]{keypad, lcd, dht},      % [4] -> orange
  %% The lines below add a nifty box around the code %%
  %frame=shadowbox,
   frame=none,  
   numbers=none,
  rulesepcolor=\color{arduinoBlue},
}

\title{Projekat iz primene senzora i aktulatora}
\author{Vladimir Galović}
\date{\today}

\begin{document}

\begin{titlepage}

\begin{center}
\includegraphics[scale=0.5]{images/ftn_logo}

\vspace{3cm}

\begin{Large}
\textbf{PROJEKAT IZ PRIMENE SENZORA I AKTUATORA}
\end{Large}
\end{center}

\vspace{1.5cm}

\begin{table}[H]
\def \arraystretch{1.25}

\setlength\parindent{16pt}
\textbf{NAZIV PROJEKTA:}\\[7pt]
\begin{tabular}{|p{16cm}|}
\hline
\setlength\parindent{10pt}
Tačan naziv projekta dogovoren sa mentorom/profesorom.\\
\hline
\end{tabular}

\vspace{0.5cm}

\textbf{TEKST ZADATKA:}\\[7pt]
\begin{tabular}{|p{16cm}|}
\hline
\setlength\parindent{10pt}
Tekst zadatka, nakon precizranja glavnih zahteva projekta u dogovoru sa mentorom/profesorom\\
\hline
\end{tabular}

\vspace{0.5cm}

\textbf{MENTOR PROJEKTA:}\\[7pt]
\begin{tabular}{|p{16cm}|}
\hline
\setlength\parindent{10pt}
Prezime i ime mentora/profesora\\
\hline
\end{tabular}

\vspace{0.5cm}

\textbf{PROJEKAT IZRADILI:}\\[7pt]
\begin{tabular}{|p{16cm}|}
\hline
\setlength\parindent{10pt}
%Prezimena, imena i brojevi indeksa  članova tima koji su radili na projektu (2-3 studenta)\\
Vladimir Galović (EE 210/2018) i Stefan Ostojić (EE 216/2016)\\
\hline
\end{tabular}

\vspace{0.5cm}

\textbf{DATUM ODBRANE PROJEKTA:}\\[7pt]
\begin{tabular}{|p{16cm}|}
\hline
\setlength\parindent{10pt}
%upisati najpribližniji poznati datum u trenutku štampanja konačne verzije\\
\today \\
\hline
\end{tabular}
\end{table}
\end{titlepage}

\tableofcontents
\pagebreak

\section{Uvod}

\vspace{10pt}

Naslov mora ostati tačno ovaj koji je napisan.

\vspace{10pt}

Obavezno navesti šta je tema projekta i ko je zadao projekat (kako je projekat smišljen, čime je motivisan, ukoliko su kandidati sami smislili projekat).

\vspace{10pt}

Poglavlje može u kratkim crtama da objasni šta predstavlja predmet projekta, čemu služi i koje su mu glavne karakteristike. Treba navesti i sve relevantne veze sa drugim oblastima od kojih će neke možda biti dotaknute u daljem tekstu.

\vspace{10pt}

Takođe je obavezno u kratkim crtama opisati ono što je napisano u pojedinim poglavljima. Npr. 
U trećem poglavlju dati su svi proračuni vezani za projekat kao i rezultati simulacija.

\vspace{10pt}

Mogu se kratko dati primeri i slike drugih uređaja iste ili slične namene. Mogu se istaći eventualne razlike koje postoje u odnosu na uređaj koji se želi razviti kao predmet ovog projekta.

\pagebreak

\section{Analiza proglema}

\vspace{10pt}

Veoma važan deo projektnog rada. Naslov se po potrebi može promeniti, ali suština teksta mora da ostane analiza problema.

\vspace{10pt}

Treba jasno istaći šta su glavni (i sporedni) ciljevi koji žele da se postignu razvojem novog uređaja. Treba navesti postojeća rešenja ukoliko postoje i mogu da se pronađu informacije o njima. Ova analiza i u praksi treba da se sprovede (ne samo u cilju navođenja u ovom tekstu) da bi se jasno zacrtao put razvoja uređaja i da bi se donele ključne odluke na osnovu kojih se neka potencijalna rešenja odbacuju, a druga prihvataju.

\vspace{10pt}

Ovakva analiza je jedan od ključnih elemenata inženjerskog razmišljanja i samim tim i jedan od ključnih ciljeva ovog projekta. Zbog toga je veoma važno da se u ovom odeljku navedu sva rešenja koja su razmatrana i da se obrazloži zašto je neko od njih usvojeno kao dobro, a neko drugo odbačeno kao neodgovarajuće za ovaj projekat.
\pagebreak

\section{Proračuni i simulacioni rezultati}

\vspace{10pt}

Ovom odeljljku se može promeniti naslov u pojedinim projektima. Za neke probleme simulacija će biti glavni izvor informacija, za neke druge pak proračun na papiru. Često se mogu i uporediti rezultati ta dva pristupa.

\vspace{10pt}

Moguće je čak i razdvojiti ovaj odeljak na više manjih ukoliko se smatra za potrebno.

\vspace{10pt}

U ovom odeljku (ili u odeljcima vezanim za njega) treba dati principske (a nekad i precizne) šeme pojedijih elemenata sistema i po potrebi dimenzionisati pojedine komponente na osnovu simulacija ili proračuna. Proračuni mogu biti obimni i često nije dobra ideja opterećivati tekst velikom količinom formula – zbog toga se sami proračuni mogu izmestiti u jedan od dodataka na kraju teksta, a u sam glavni tekst se tada dodaje samo referenca (poziv) na odgovarajući dodatak.

\vspace{10pt}

Same formule ukoliko ih ima treba da budu numerisane kao u primeru ispod:

\begin{align}
I_{CQ}= \frac{ V_{CC}-V_T-V_{BE} }{ \left(R_B+R_E h_{FE}+1 \right) }h_{FE}  \label{Icq}
\end{align}

\vspace{10pt}

Preko broja jednačine može se pozivati na određenu jednačinu u tekstu, npr vidi jednačinu broj \ref{Icq}.

\vspace{10pt}

Simulacioni rezultati mogu da potiču iz različitih softvera. Mogu to biti simulatori električnih kola (DC, AC, tranzijentne, Monte Carlo analize), a i simulatori opšteg tipa ili neki drugi softverski inženjerski alat.

\pagebreak

\section{Opis detalja predmeta projekta}

\vspace{10pt}
	
	\subsection{Detaljan opis svih podsistema uređaja}

\vspace{10pt}
	
	Ne treba opisivati pojedinačne komponente ukoliko su one široko poznate (poput otpornika, kondenzatora, tranzistora, operacionih pojačavača). Ako je pak neki od njih poseban i po svojim karakteristikama ključan za rad uređaja, treba ga adekvatno pomenuti. Specijalne senzore i integrisana kola treba pomenuti posebno jer njihovi detalji obično nisu svakom poznati.

\vspace{10pt}

Važno je ovaj deo ne preopterećivati velikim šemama nego šeme podsklopova uređaja treba davati pojedinačno u delovima iz kojih je jasan njihov način rada. Poželjno je dati blok šemu kompletnog uređaja iz kog je jasan princip rada sistema i na osnovu kog se mogu objasniti uloge pojedinačno navedenih podsklopova.

\vspace{10pt}

Celokupnu šemu kao i dokumentaciju bitnih komponenti treba dati u dodacima (na kraju rada), a u ovom odeljku treba samo skrenuti pažnju čitaocu (upućivati ga na dodatke i literaturu) gde će naći detaljnije informacije.

\vspace{10pt}

	\subsection{Slika uređaja u krajnjem stadijumu izrade}

\vspace{10pt}

Poželjno je u tekst staviti barem jednu sliku gotovog uređaja, čak i ako je on konačno izrađen samo na protobordu. To treba da bude što je moguće reprezentativnija slika koja na očigledan način pokazuje i dokazuje da je uređaj zaista napravljen i da se po mogućstvu vidi i njegova funkcija i način rada.

\pagebreak

\section{Rezultati testiranja}

\vspace{10pt}

Nakon što je uređaj izrađen može se pristupiti njegovom testiranju. Testiranje podrazumeva proveru performansi u praksi.

\vspace{10pt}

Testiranje treba da bude primereno datom konkretnom uređaju. Kakvo će testiranje biti primereno treba utvrditi konsultacijom odgovarajuće literature i dogovorom sa mentorom. Na primer, za pojačavač je merodavno koliku snagu može da isporuči bez vidnih izobličenja, za izvor napajanja je bitan opseg izlaznog napona i strujno ograničenje. Za složenije sisteme može biti potrebno da se osmisli celokupan algoritam provere funkcionalnosti.

\vspace{10pt}

Treba proveriti i u ovom odeljku opisati koliko uređaj uspešno obavlja zadatak za koji je projektovan. To treba potkrepiti merenjima ulaznih test signala, izlaznih signala i unutrašnjih signala koji predstavljaju nekakav međurezultat. Mogu se dati tabele sa rezultatima, frekvencijske karakteristike, slike dobijene pomoću oscilokopa i slično ukoliko to može da koristi u predočavanju performansi.
\pagebreak

\section{Zaključak}

\vspace{10pt}


Ovaj odeljak je obavezan. Naslov takođe mora ostati nepromenjen. Možemo ga podeliti na dva dela:

\begin{itemize}
	\item Zaključak o uspešnosti i meri završenosti projekta,
	\item Rezime projekta.
\end{itemize}

\vspace{10pt}

Veoma je važno na samom početku ovog dela jasno reći da li je urađeno sve ono što je bilo definisano projektnim zadatkom. Ako ima odstupanja od toga, onda kratko objasniti koja su to odstupanja i zašto su nastala. Ako su nastala odstupanja od definisanog projekta, napisati u kom mestu u radu mogu da se nađu detaljnije informacije o tome.

\vspace{10pt}

Rezime je mesto gde se pravi osvrt na ono što je urađeno u projektu, ili naučnom radu. Dužina rezimea može da bude od dva do tri pasusa. Prvi pasus je obavezan i predstavlja pregled onoga što je urađeno. U njemu se u kratkim crtama nabroji ono što je opisano u tekstu koji predhodi zaključku. Npr. može se sa po jednom rečenicom ponoviti suština ili rezulat svakog odeljka koji je napisan pre zaključka. Ovaj pasus je obavezan.

\vspace{10pt}

Drugi pasus može ukratko da ponovi ono što je nesumnjivi zaključak celog rada i koji treba da bude posebno istaknut kao vredan rezultat. Ovaj pasus nije obavezan.

\vspace{10pt}

Treći pasus je obavezan i predstavlja mesto gde mogu ukratko da se navedu ideje koje su se javile tokom izrade projekta ili pisanja rada i predstavljaju dalje moguće pravce razvoja konkretnog uređaja ili nove oblasti koje autor rada namerava u bliskoj budućnosti da istraži.
\pagebreak

\section{Literatura}

\vspace{10pt}


Ovaj odeljak je obavezan. Ne treba da sadrži nikakav drugi tekst osim spiska literature. Literatura može da bude stručna knjiga, udžbenik, zbirka zadataka, praktikum, standard, članak iz časopisa, naučni rad, web-sajt (daje se poslednji važeći link na njega), tehnička dokumentacija (manual ili datasheet). Spisak se daje kao numerisana lista gde su za svaku stavku dati relevantni podaci ukoliko su poznati. Tu spadaju: autor(i), naslov, izdavač, godina izdanja, stranice u slučaju časopisa ili zbornika radova itd. Ovaj tekst obavezno obrisati. Na kraju odeljka dat je primer spiska literature.

\vspace{10pt}

Svrha navođenja literature je da se čitoaocu ukaže na mesta gde može naći detaljnije informacije o pojedinostima koje se pominju u tekstu. Takođe, reference ističu da se ovaj rad nadovezuje na neka od ranije poznata znanja. Uobičajeno je da se kategoričke tvrdnje koje se ne dokazuju u ovom radu potkrepljuju navođenjem literature u kojoj je to obrazloženo ili dokazano. U takvim slučajevima se neposredno iza navedene tvrdnje navodi oznaka literature, najčešće ka broj u uglastim zagradama npr. [2] predstavlja pozivanje na knjigu "Infrared Sensors" iz primera literature.


\begin{enumerate}[label={[1]},leftmargin=2.5cm]
	\item Ime Prezime Autora, Kuvar elektronike, FTN, Novi Sad, 2008.
	\item T. K. Webber, Infrared Sensors, MCMillan, 2002.
\end{enumerate}
\pagebreak

\section*{Dodatak A}
	\addcontentsline{toc}{section}{Dodatak A}

\vspace{10pt}


Dodaci su obično obeleženi slovima i tako se na njih i poziva. Poželjno je takođe da svaki dodatak počne na novoj strani.

\vspace{10pt}

U dodatke mogu da se stave i najvažniji dokumenti koji se odnose na pojedine komponente i podsklopove (kupljenje kao gotove celine) – tzv. Datasheet. 

\vspace{10pt}

\begin{lstlisting}[style=Arduino]
#include <SD.h>

File logfile;
byte logPin = 10;

void setup() {
  SD.begin(logPin);

  ///////// Create a new file //////////
  char filename[] = "LOGGER00.CSV";
  for (int i = 0; i < 100; i++) {
    filename[6] = i/10 + '0';    // number the file
    filename[7] = i%10 + '0';    //
    if ( SD.exists(filename)==false ) { // only open a new file if it doesn't exist
      logfile = SD.open(filename, FILE_WRITE);
      break;  // leave the loop!
    }
  }
}

void loop() {
  // put your main code here, to run repeatedly:
  logfile.println("Hello");
  logfile.flush();
  delay(1000);
}
\end{lstlisting}
\end{document}